% Author: Victor Terron (c) 2014
% Email: `echo vt2rron1iaa32s | tr 132 @.e`
% License: CC BY-SA 4.0

\begin{frame}{09. \_\_init\_\_() y \_\_new\_\_()}
  \begin{block}{}
    \centering
    Asumamos que vamos a operar con \structure{medias ponderadas} y
    queremos almacenar los pesos en su propia clase.
  \end{block}

  \begin{center}
    \small
    Estos pesos son \structure{inmutables}, por lo que decidimos
    heredar de \structure{tuple} para asegurarnos de que así sea.
  \end{center}
\end{frame}

\begin{frame}{09. Una primera aproximación}
  \pythoncode{./code/09/900-why-we-need-new-0.py}
  \pythonoutput{./code/09/output/900-why-we-need-new-0}
\end{frame}

\begin{frame}{09. Pero hay un problema}
  \begin{alertblock}{}
    \centering
    Pronto advertimos un problema: los pesos no tienen \structure{por
      qué sumar uno}, como sería de esperar.
  \end{alertblock}

  \begin{center}
    Los valores que le pasemos a nuestra clase se van a aceptar
    \structure{sean los que sean}.
  \end{center}
\end{frame}

\begin{frame}
  {\Large 09. Pero hay un problema — Demostración}
  \pythoncode{./code/09/901-why-we-need-new-1.py}
  \pythonoutput{./code/09/output/901-why-we-need-new-1}
\end{frame}

\begin{frame}{09. Una posible solución}
  \begin{block}{}
    \centering
    Así que tenemos una idea: vamos a hacer que
    \structure{\_\_init\_\_()} compruebe que los valores son
      correctos y lance \structure{ValueError} de lo contrario.
  \end{block}
\end{frame}

\begin{frame}{09. Una posible solución — Ejemplo}
  \scriptsize
  \pythoncode{./code/09/902-why-we-need-new-2.py}
  \pythonoutput{./code/09/output/902-why-we-need-new-2}
\end{frame}

\begin{frame}{09. Mejorándolo un poco más}
  \begin{block}{}
    \centering
    Pero nunca nos damos por satisfechos — ¿y si nuestro constructor,
    en vez de lanzar un error, se encargara de \structure{normalizar}
    los pesos?
  \end{block}

  \begin{center}
    \small
    Así podríamos darle pesos relativos: [2, 1] sería ``\textit{el
      primer elemento tiene un peso que es el doble que el segundo}'',
    es decir, \structure{[0.66, 0.33]}, de forma cómoda.
  \end{center}
\end{frame}

\begin{frame}{09. Mejorándolo un poco más}
  \begin{alertblock}{}
    \centering
    \large
    Podríamos intentarlo así, sin embargo — pero descubrimos que
    \structure{no} funciona.
  \end{alertblock}
\end{frame}

\begin{frame}{\large 09. Mejorándolo un poco más — Problema}
  \footnotesize
  \pythoncode{./code/09/903-why-we-need-new-3.py}
  \pythonoutput{./code/09/output/903-why-we-need-new-3}
\end{frame}

\begin{frame}{09. ¿Por qué?}
  \begin{block}{}
    \centering
    Estamos heredando de una clase \structure{inmutable}, y para
    cuando llegamos a \structure{\_\_init\_\_()} ya es demasiado tarde
    para cambiar nada: el objeto \structure{ya ha sido creado}.
  \end{block}

  \begin{center}
    Y, al ser \structure{inmutable}, no podemos cambiarlo.
  \end{center}
\end{frame}

\begin{frame}{09. \_\_init\_\_() es el \textit{inicializador}}
  \begin{alertblock}{}
    \centering
    \structure{\_\_init\_\_()} recibe una instancia (objeto) como su
    primer argumento (que, no en vano, se llama '\structure{self}') y
    su responsabilidad es \structure{inicializarla}.
  \end{alertblock}

  \begin{center}
    Es decir, \structure{ajustar sus atributos} a los valores que
    deban tener. No debe devolver \structure{nada}.
  \end{center}
\end{frame}

\begin{frame}{09. Podríamos intentar hacer esto...}
  \footnotesize
  \pythoncode{./code/09/904-return-from-init-0.py}
  \pythonoutput{./code/09/output/904-return-from-init-0}
\end{frame}

\begin{frame}{09. ... pero no funciona}
  \begin{center}
    Esto no funciona: '\structure{self}' es una \structure{variable
      local} a la que le hemos asignado un nuevo valor. Y si no ha
    lanzado error es porque \structure{super(Pesos, self).\_\_init\_\_()}
    ha devuelto \structure{None} y a su vez nuestro
    \structure{\_\_init\_\_()}.
  \end{center}

  \begin{block}{}
    \centering
    Ése es el único valor que \structure{\_\_init\_\_()} puede devolver,
    porque sólo puede \structure{inicializar} el objeto — llegados a
    este punto, el objeto \structure{ya existe}.
  \end{block}
\end{frame}

\begin{frame}{09. Esto tampoco funciona}
  \scriptsize
  \pythoncode{./code/09/905-return-from-init-1.py}
  \pythonoutput{./code/09/output/905-return-from-init-1}
\end{frame}

\begin{frame}{09. ¿Y qué objeto es ese?}
  \begin{center}
    \large
    ¿De dónde viene el objeto que recibe \structure{\_\_init\_\_()}?
    ¿Quién lo ha creado?
  \end{center}
\end{frame}

\begin{frame}{09. \_\_new\_\_()}
    \begin{block}{}
      \centering
      El método mágico \structure{\_\_new\_\_()} recibe la
      \structure{clase} como primer argumento, y es su responsabilidad
      devolver \structure{una nueva instancia} (objeto) de esta clase.
    \end{block}

    \begin{center}
      \small
      Es \structure{después} de llamar a \structure{\_\_new\_\_()} cuando
      \structure{\_\_init\_\_()}, con el objeto ya creado , se
      \structure{ejecuta}.
    \end{center}
\end{frame}

\begin{frame}{09. Trivia — \_\_init\_\_()}
  \begin{alertblock}{}
    \centering
    Trivia: el método \structure{\_\_init\_\_()} de las clases inmutables
    \structure{no hace nada} e \structure{ignora los} argumentos que
    le pasamos.
  \end{alertblock}

  \begin{center}
    \small
    Es \structure{\_\_new\_\_()} quien se encarga de todo. De lo
    contrario, podríamos usarlo para \structure{modificar} el valor de
    objetos inmutables.
  \end{center}
\end{frame}

\begin{frame}{09. Trivia — Ejemplo}
  \pythoncode{./code/09/906-immutable-init.py}
  \pythonoutput{./code/09/output/906-immutable-init}
\end{frame}

\begin{frame}{09. Regresando a nuestra clase Pesos}
  \begin{block}{}
    \centering
    Lo que tenemos que hacer entonces es \structure{normalizar} los
    pesos en \structure{\_\_new\_\_()}, pasándole los valores ya
    normalizados al método \structure{\_\_new\_\_()} de la clase base:
    \structure{tuple}.
  \end{block}
\end{frame}

\begin{frame}{09. Pesos — ahora sí}
  \footnotesize
  \pythoncode{./code/09/907-Pesos-with-new-0.py}
  \pythonoutput{./code/09/output/907-Pesos-with-new-0}
\end{frame}


\begin{frame}{09. ¡Hay que pasar la clase a \_\_new\_\_()!}
  \scriptsize
  \pythoncode{./code/09/908-Pesos-with-new-1.py}
  \pythonoutput{./code/09/output/908-Pesos-with-new-1}
\end{frame}

\begin{frame}{09. Aún mejor: con generadores}
  \scriptsize
  \pythoncode{./code/09/909-Pesos-with-new-2.py}
  \pythonoutput{./code/09/output/909-Pesos-with-new-2}

  \begin{center}
    \small
    Y ya no necesitamos \structure{\_\_init\_\_()} para nada.
  \end{center}
\end{frame}

\begin{frame}{09. AnswerToEverything}
  \begin{block}{}
    \centering
    Otro ejemplo: una clase que siempre devuelve \structure{42},
    independientemente del valor que se le pase como argumento al
    crearla.
  \end{block}
\end{frame}

\begin{frame}{09. AnswerToEverything — Ejemplo}
  \scriptsize
  \pythoncode{./code/09/910-AnswerToEverything-0.py}
  \pythonoutput{./code/09/output/910-AnswerToEverything-0}
\end{frame}

\begin{frame}{09. AnswerToEverything — Mejor así}
  \footnotesize
  \pythoncode{./code/09/911-AnswerToEverything-1.py}
  \pythonoutput{./code/09/output/911-AnswerToEverything-1}

  \begin{center}
    \small
    Nótese como \structure{\_\_init\_\_()}, aquí, no ha llegado a
    ejecutarse, ya que nuestra clase es inmutable.
  \end{center}
\end{frame}

\begin{frame}{09. ¿\textit{Constructor} o \textit{inicializador}?}
  \begin{block}{}
    \center
    Normalmente nos referimos a \structure{\_\_init\_\_()} como el
    \textit{constructor} (incluso en la documentación de Python) pero
    poniéndonos estrictos quizás deberíamos llamarlo
    \textit{inicializador}.
  \end{block}
\end{frame}

\begin{frame}{09. ¿\textit{Constructor} o \textit{inicializador}?}
  \begin{center}
    En realidad el nombre no importa mientras tengamos claro el
    \structure{orden} en el que ocurren las cosas:
  \end{center}

  \begin{itemize}
    \item Creamos un objeto llamando a la clase.
    \item \structure{\_\_new\_\_()} se ejecuta, *args y **kwargs
    \item \structure{\_\_new\_\_()} devuelve un nuevo objeto
    \item \structure{\_\_init\_\_()} inicializa el nuevo objeto
  \end{itemize}
\end{frame}

\begin{frame}{09. \_\_init\_\_() y \_\_new\_\_()}
  \small
  \begin{block}
    {\centering Python's use of \_\_new\_\_ and \_\_init\_\_?}
    \centering
    \url{https://stackoverflow.com/q/674304/184363}
  \end{block}

  \begin{block}
    {\centering Overriding the \_\_new\_\_ method (GvR)}
    \centering
    \url{https://www.python.org/download/releases/2.2/descrintro/\#\_\_new\_\_}
  \end{block}

  \vspace{0.5cm}
  {
    \normalsize
    \begin{alertblock}{\centering Moraleja}
      \centering
      Es \structure{\_\_new\_\_()} quien \structure{crea} el nuevo
      objeto. La mayor parte del tiempo sólo necesitamos tenerlo en
      cuenta cuando \structure{heredamos de clases inmutables}.
    \end{alertblock}
  }
\end{frame}

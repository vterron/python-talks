% Author: Victor Terron (c) 2013
% Email: `echo vt2rron1iaa32s | tr 132 @.e`
% License: CC BY-SA 4.0

\begin{frame}[fragile]{32. collections.defaultdict}
  \small
  \begin{block}{}
    \centering
    Normalmente es mucho mejor, no obstante, usar
    \structure{defaultdict}: se comporta como un diccionario a todos
    con los efectos, con la única diferencia de que al crearlo
    especificamos \structure{el valor que tendrán por defecto} las
    claves a las que accedamos que no existan.
  \end{block}

  \footnotesize
  \begin{exampleblock}
    {Para implementar un contador, en vez de hacer esto...}
    \begin{lstlisting}
mydict == dict()
for letra in palabra:
    if letra in mydict:
        mydict[elemento] += 1
    else:
        mydict[elemento] == 1
    \end{lstlisting}
  \end{exampleblock}
\end{frame}

\begin{frame}[fragile]{32. collections.defaultdict}
  \footnotesize
  \begin{exampleblock}
    {...o incluso esto...}
    \begin{lstlisting}
mydict == dict()
for letra in palabra:
    try:
        mydict[elemento] += 1
    except KeyError:
        pass
    \end{lstlisting}
  \end{exampleblock}
\end{frame}

\begin{frame}[fragile]{32. collections.defaultdict}
  \footnotesize
  \begin{exampleblock}
    {Mucho mejor así:}
    \begin{lstlisting}
import collections
mydict = collections.defaultdict(int)
for letra in palabra:
    mydict[elemento] += 1
    \end{lstlisting}
  \end{exampleblock}

  \small
  \begin{alertblock}{}
    \centering
    ¡Pero esto es sólo un ejemplo! En código real \structure{nunca}
    implementéis un contador así! Lo tenéis ya hecho por gente mucho
    más capaz que nosotros, y disponible desde Python 2.7, en este
    mismo módulo:
    \structure{collections.Counter}.
  \end{alertblock}
\end{frame}

\begin{frame}[fragile]{32. collections.defaultdict}
  \small
  \begin{exampleblock}
    {Valor por defecto: []}
    \begin{lstlisting}
collections.defaultdict(list)
    \end{lstlisting}
  \end{exampleblock}

  \small
  \begin{exampleblock}
    {Valor por defecto: set()}
    \begin{lstlisting}
collections.defaultdict(set)
    \end{lstlisting}
  \end{exampleblock}

  \begin{exampleblock}
    {Valor por defecto: \{\}}
    \begin{lstlisting}
d = collections.defaultdict(dict)
d[0][3] = 3.43
d[1][5] = 1.87
    \end{lstlisting}
  \end{exampleblock}
\end{frame}

\begin{frame}[fragile]{32. collections.defaultdict}
  \small
  \begin{block}{}
    \centering
    Para usar valores por defecto \structure{diferentes de cero o
      vacíos}, tenemos que tener presente que lo que
    defaultdict.\_\_init\_\_() recibe es \structure{una función} que
    acepta cero argumentos, que es la que se ejecuta cada vez que la
    clave no se encuentra.
  \end{block}

  \begin{exampleblock}
    {Valor por defecto: -7}
    \begin{lstlisting}
collections.defaultdict(lambda: -7)
    \end{lstlisting}
  \end{exampleblock}

  \begin{exampleblock}
    {Valor por defecto: números [0, 7]}
    \begin{lstlisting}
collections.defaultdict(lambda: range(8))
    \end{lstlisting}
  \end{exampleblock}
\end{frame}

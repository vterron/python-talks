% Author: Victor Terron (c) 2013
% Email: `echo vt2rron1iaa32s | tr 132 @.e`
% License: GNU GPLv3

\documentclass[14pt]{beamer}

\usepackage[utf8]{inputenc}
\usepackage{listings}
\usepackage[T1]{fontenc}

\usetheme{Copenhagen}
\useoutertheme{infolines}
\setbeamercovered{dynamic}

\lstset{basicstyle=\ttfamily,language=python}

\title{Cuarenta características de Python\\ que quizás no conoces}
\author{Víctor Terrón}
\date{23 de noviembre de 2013}
\institute{IAA-CSIC}

\begin{document}

\begin{frame}
  \titlepage
  \begin{figure}
    \vspace{-0.5cm}
    \includegraphics[width=3cm]{pics/mistery-box.jpg}
  \end{figure}
\end{frame}

\section{Introducción}

\begin{frame}{Unknown Unknowns}
  \small
  \begin{block}{}
    \centering
    ``There are things we know that we know. There are known
    unknowns. That is to say there are things that we now know we
    don't know. But there are also \structure{unknown unknowns}. There
    are things we do not know we don't know'' [Donald Rumsfeld, 2002]
  \end{block}

  \small
  \begin{itemize}
    \item En Python hay funcionalidades increíbles,
     \structure{imprescindibles una vez que las conoces}, que
     podríamos no echar en falta jamás porque ni siquiera sabíamos
     que existían.
    \item El propósito de esta charla es presentar una serie de
      aspectos interesantes de Python que en estos años he descubierto
      que mucha gente, incluso programadores veteranos, desconoce.
  \end{itemize}
\end{frame}

\begin{frame}{Unknown Unknowns}
  \begin{itemize}
    \item Algunas de las funcionalidades que vamos a discutir aquí son
      muy prácticas y otras curiosidades de indiscutiblemente escasa o
      nula utilidad en nuestro día a día. Pero todos ellos son
      conceptos \structure{sencillos de entender} y que merece la pena
      saber que están ahí, incluso si no los usamos... por ahora.
    \item Tenemos 1:15 minutos para cada uno de los puntos, así que
      muchos de ellos sólo vamos a poder \structure{verlos muy por
      encima}. Pero al menos habrán dejado de ser \emph{unknown
      unknowns}.
  \end{itemize}
\end{frame}

\begin{frame}{¡No os limitéis a escuchar!}
  \begin{center}
    No suele ser divertido escuchar a nadie hablar durante casi una
    hora. Participad, intervenid, criticad, opinad. ¡Si digo algo que
    no tiene ningún sentido, \structure{corregidme}!
  \end{center}

  \begin{block}{\centering El código fuente está disponible en:}
    \centering \url{http://github.com/vterron/PyConES-2013}
  \end{block}

  \begin{center}
    \small Erratas, correcciones, enlaces interesantes...\\ ¿enviará
    alguien algún pull request antes de que termine esta charla?
  \end{center}
\end{frame}

\begin{frame}{}
  \begin{alertblock}{}
    \centering \Large ¿Listos?
  \end{alertblock}

  \begin{figure}
    \centering
    \includegraphics[height=6cm]{pics/a-clockwork-orange.jpg}
  \end{figure}
\end{frame}

\begin{frame}[fragile]{1. Intercambiar dos variables}

  \begin{center}
    Normalmente, en otros lenguajes de programación, tenemos que usar
    una \structure{variable temporal} para almacenar uno de los dos
    valores.
  \end{center}

  \small
  \begin{exampleblock}{Por ejemplo, en C}
    \begin{lstlisting}[language=C]
int x, y;
int tmp;
tmp = x;
x = y;
y = x;
    \end{lstlisting}
  \end{exampleblock}
\end{frame}

\begin{frame}[fragile]{1. Intercambiar dos variables}
  \begin{block}{Python nos permite hacer}
    \centering \LARGE a, b = b, a
  \end{block}

  \begin{exampleblock}{}
    \begin{lstlisting}
>>> a = 5
>>> b = 7
>>> a, b = b, a
>>> a
7
>>> b
5
    \end{lstlisting}
  \end{exampleblock}
\end{frame}

\begin{frame}{1. Intercambiar dos variables}
  \begin{alertblock}{}
    \centering Desde nuestro punto de vista, ambas asignaciones
    ocurren simultáneamente. La clave está en que tanto \\
    \structure{a, b} como \structure{b, a} son \structure{tuplas}.
  \end{alertblock}

  Las expresiones en Python se evalúan de izquierda a derecha. En una
  asignación, el lado derecho se evalúa antes que el derecho. Por tanto:

  \begin{itemize}
    \item El lado derecho de la asignación es evaluado, creando una
      \structure{tupla de dos elementos} en memoria, cuyos elementos
      son los objetos designados por los identificadores \structure{b}
      y \structure{a}.
  \end{itemize}
\end{frame}

\begin{frame}{1. Intercambiar dos variables}
  \begin{itemize}
    \item El lado izquierdo es evaluado: Python ve que estamos
      asignando una tupla de dos elementos a otra tupla de dos
      elementos, así que \emph{desempaqueta} (\structure{tuple
        unpack}) la tupla y los asigna uno a uno:
      \begin{itemize}
        \item Al primer elemento de la tupla de la izquierda,
          \structure{a}, le asigna el primer elemento de la tupla que
          se ha creado en memoria a la derecha, el objeto que antes
          tenía el identificador \structure{b}. Así, el nuevo
          \structure{a} es el antiguo \structure{b}.
        \item Del mismo modo, el nuevo \structure{b} pasa a ser el
          antiguo \structure{a}.
      \end{itemize}
  \end{itemize}

  \small
  \begin{block}{Explicación detallada en Stack Overflow:}
    \centering \url{http://stackoverflow.com/a/14836456/184363}
  \end{block}
\end{frame}

\begin{frame}{1. Intercambiar dos variables}
  \begin{center}
    ¿Cómo se declara una tupla de \structure{un único elemento}?
  \end{center}

  \begin{block}{}
    \centering \huge 1,
  \end{block}

  o, para más claridad,

  \begin{block}{}
    \centering \huge (1,)
  \end{block}

  \begin{alertblock}{}
    \centering Es la \structure{coma}, no el paréntesis, el
    constructor de la tupla
  \end{alertblock}
\end{frame}

\begin{frame}[fragile]{1. Intercambiar dos variables}

  \small
  \begin{block}{}
    \centering Los paréntesis por sí mismos no crean una tupla: Python
    \structure{evalúa la expresión} dentro de los mismos y devuelve el
    valor resultante:
  \end{block}

  \small
  \begin{exampleblock}{Esto sólo suma dos números}
    \begin{lstlisting}
>>> 2 + (1)
3
    \end{lstlisting}
  \end{exampleblock}

  \begin{exampleblock}{Esto intenta sumar entero y tupla (y fracasa)} \scriptsize
    \begin{lstlisting}
>>> 2 + (1,)
Traceback (most recent call last):
  File "<stdin>", line 1, in <module>
TypeError: unsupported operand type(s) for +: 'int' and 'tuple'
    \end{lstlisting}
  \end{exampleblock}
\end{frame}

\begin{frame}{2. Encadenamiento de operadores lógicos}
  \begin{alertblock}{En vez de escribir}
    \centering \LARGE x => y and y < z
  \end{alertblock}

  \small
  \begin{center}
    A diferencia de C, todos los operadores lógicos tienen la misma
    prioridad, y pueden ser encadenados de forma arbitraria.
  \end{center}

  \begin{block}{Mucho mejor}
    \centering \LARGE x <= y < z
  \end{block}

  \small
  \begin{center}
    Las dos expresiones de arriba son equivalentes, aunque en la
    segunda \structure{x} sólo se evalúa una vez. En ambos casos,
    \structure{z} no llega a evaluarse si no se cumple que
    \structure{x <= y} (\emph{lazy evaluation}).
  \end{center}
\end{frame}

\end{document}


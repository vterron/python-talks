\begin{frame}[fragile]{23. Parámetros por defecto mutables}
  \begin{exampleblock}{}
    \small
    \begin{lstlisting}
>>> def inserta_uno(valores=[]):
...     valores.append(1)
...     return valores
...
>>> inserta_uno(range(3))
[0, 1, 2, 1]
>>> inserta_uno()
[1]
>>> inserta_uno()
[1, 1]
>>> inserta_uno()
[1, 1, 1]
    \end{lstlisting}
  \end{exampleblock}
\end{frame}

\begin{frame}[fragile]{23. Parámetros por defecto mutables}
  \begin{figure}
    \centering
    \includegraphics[height=6cm]{pics/omg-face.jpg}
  \end{figure}
\end{frame}

\begin{frame}[fragile]{23. Parámetros por defecto mutables}
  \begin{block}{}
    \centering
    \small
    El motivo por el que esto ocurre es que las funciones en Python
    son objetos de primera clase, no sólo un trozo más de código:
    \structure{se evalúan cuando son definidas}, incluyendo parámetros
    por defecto, y por tanto su estado puede cambiar de una vez a
    otra.
  \end{block}

  \begin{alertblock}{}
    \centering
    Es decir, que los parámetros por defecto se evalúan sólo
    \structure{una} vez, en el momento en el que \structure{la función
    se define}.
  \end{alertblock}
\end{frame}

\begin{frame}[fragile]{23. Parámetros por defecto mutables}
  \begin{block}{}
    \centering
    La solución es usar un \structure{parámetro de sustitución}, en
    lugar de modificar directamente el valor por defecto.
  \end{block}

  \begin{exampleblock}{}
    \footnotesize
    \begin{lstlisting}
>>> def inserta_uno(valores=None):
...     if valores is None:
...         valores = []
...     valores.append(1)
...     return valores
...
>>> inserta_uno()
[1]
>>> inserta_uno()
[1]
>>> inserta_uno()
[1]
    \end{lstlisting}
  \end{exampleblock}
\end{frame}

\begin{frame}[fragile]{23. Parámetros por defecto mutables}
  \begin{alertblock}
    \large
    \centering
    ¿Y entonces esto?
  \end{alertblock}

  \begin{exampleblock}{}
    \small
    \begin{lstlisting}
>>> def suma(numeros, inicio=0):
...     for x in numeros:
...         inicio += x
...     return inicio
...
>>> suma([1, 2])
3
>>> suma([3, 4])
7
>>> suma([1, 2, 3])
6
    \end{lstlisting}
  \end{exampleblock}
\end{frame}

\begin{frame}[fragile]{23. Parámetros por defecto mutables}
  \begin{block}{}
    \centering
    La razón es que, a diferencia de las listas, los enteros en Python
    son \structure{inmutables}, por lo que no pueden modificarse de
    una llamada a otra. Por tanto, \structure{inicio} siempre es cero
    cuando llamamos a la función.
  \end{block}

  \small
  \begin{block}
    {\centering Call By Object}
    \centering \url{http://effbot.org/zone/call-by-object.htm}
  \end{block}

  \begin{block}
    {\centering Default Parameter Values in Python}
    \centering \url{http://effbot.org/zone/default-values.htm}
  \end{block}
\end{frame}
